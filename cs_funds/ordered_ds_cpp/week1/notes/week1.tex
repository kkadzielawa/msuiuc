\documentclass{article}
\usepackage{color}
\usepackage[top=1in, bottom=1in, left=1in, right=1in]{geometry}
\usepackage{amsmath}
\usepackage[utf8]{inputenc}
\usepackage{listings}
\usepackage{xcolor}
\usepackage{sansmath}
\usepackage{graphicx}
\graphicspath{ {./images/} }
\usepackage{float}
\usepackage{mathtools}

\begin{document}

\section{Information}




\section{Lectures - Linear Structures}


\subsection{Arrays}

\begin{itemize}
	\item 
\end{itemize}

























































\section{Readings}




\section{Graded Activities}







\iffalse

\section{Initial Set Up}
-install python
-create dedicated virtual environment
-install django

\subsection{Install Python 3 on Windows}
- install at least python 3.10
\begin{figure}[H]
	\centering
	\includegraphics[scale=.75]{fig1_1}
	\caption {Check Python Version in CMD}
\end{figure}


\subsection{Virtual Environments}
-by default python and django are installed \textit{globally} on the computer
-what if two projects use two different Django versions?
-you should use a dedicated virtual environment for each new Python project
-simplest way is with \textbf{venv}
\begin{figure}[H]
	\centering
	\includegraphics[scale=.75]{fig1_2}
	\caption {Using venv}
\end{figure}
-.venv is a common choice for environment name
-once created, you need to \textit{activate} the virtual environment using
\begin{figure}[H]
	\centering
	\includegraphics[scale=.75]{fig1_3}
	\caption {Activating venv}
\end{figure}
- to deactivate, type \textit{deactivate} into CMD


\subsection{Install Django}

-activate virtual environment first
-Django is hosted on PyPI - central repo for most python packages
-use \textit{pip} to install it 
-``\~=" ensures subsequent security updates are automatically installed
\begin{figure}[H]
	\centering
	\includegraphics[scale=.75]{fig1_4}
	\caption {Installing Django}
\end{figure}



\subsection{First Django Project}
- to create a new Django project use command
\begin{figure}[H]
	\centering
	\includegraphics[scale=.75]{fig1_5}
	\caption {Starting Django Project}
\end{figure}

\begin{figure}[H]
	\centering
	\includegraphics[scale=.75]{fig1_6}
	\caption {Default Django Directory Structure}
\end{figure}

- you add . at the end to avoid the duplicate structure of project name being nested

\subsection{Text Editors}
- add python and black python formatter extensions


\subsection{Install Git}
\begin{figure}[H]
	\centering
	\includegraphics[scale=.75]{fig1_7}
	\caption {One Time Git Configs}
\end{figure}

\fi

\end{document}

